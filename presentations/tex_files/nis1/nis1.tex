\documentclass{beamer}
%
% Choose how your presentation looks.
%
% For more themes, color themes and font themes, see:
% http://deic.uab.es/~iblanes/beamer_gallery/index_by_theme.html
%
\mode<presentation>
{
  \usetheme{Madrid}      % or try Darmstadt, Madrid, Warsaw, ...
  \usecolortheme{beaver} % or try albatross, beaver, crane, ...
  \usefonttheme{serif}  % or try serif, structurebold, ...
  \setbeamertemplate{navigation symbols}{}
  \setbeamertemplate{caption}[numbered]
} 

%%% Работа с русским языком
\usepackage{cmap}					% поиск в PDF
\usepackage[T2A]{fontenc}			% кодировка
\usepackage[utf8]{inputenc}			% кодировка исходного текста
\usepackage[english,portuguese, main=russian]{babel}	% локализация и переносы

%Диаграмы
\usepackage{smartdiagram}

\usepackage[english]{babel}
%\usepackage[utf8x]{inputenc}
%\usepackage{xcolor}
\usepackage{listings}
\lstset
{
    language=[LaTeX]TeX,
    breaklines=true,
    basicstyle=\tt\scriptsize,
    %commentstyle=\color{green}
    keywordstyle=\color{blue},
    %stringstyle=\color{black}
    identifierstyle=\color{magenta},
}

\title[Научно-исследовательский семинар]{Комплинг исследование научно-популярных текстов}
\author{Анастасия Кузнецова, Анна Лапидус, \\ Юлия Коломенская, Ксения Самойленко}
\institute{НИУ ВШЭ}
\date{9 ноября 2017}

\AtBeginSection[]
{
  \begin{frame}<beamer>
    \frametitle{План доклада}
    \tableofcontents[currentsection,currentsubsection]
  \end{frame}
}

\begin{document}

\begin{frame}
  \titlepage
\end{frame}

% Uncomment these lines for an automatically generated outline.
\begin{frame}{План доклада}
  \tableofcontents
\end{frame}

\section{Цели и задачи проекта}

\begin{frame}{Цели и задачи проекта}
\textbf{Цель проекта:} \\
Компьютерно-лингвистический анализ научно-популярных текстов на русском языке.\\
\textbf{Ожидаемые результаты проекта:} \\
\begin{itemize}
\item Создание словаря научно-популярных терминов и рейтинга цитируемости ученых
\item Написание нескольких статей на основе выводов, сделанных в ходе исследования
\item Потенциал для использования полученных данных в прикладной сфере, например, для создания компилятора научно-популярных новостей или обзоров
\end{itemize}

\end{frame}

\begin{frame}{Цели и задачи проекта}
\textbf{Задачи для реализации проекта:} \\
\begin{itemize}
\item Проанализировать список русскоязычных научно-популярных ресурсов и определить релевантные для данного исследования
\item Собрать данные:
	\begin{itemize}
	\item Запросить непосредственно у владельцев ресурсов
    \item Обкачать сайты (учимся работать с краулером, API)
	\end{itemize}
\item Разметить и структурировать полученные данные (text clusterization)
\item Непосредственно анализ данных (text similarities, fact extraction etc)
\item Выводы
\end{itemize}

\end{frame}

\section{Этапы работы}

\begin{frame}{Этапы работы}
\smartdiagram[descriptive diagram]{
  {Модуль 2,{\textbf{(0--1)} Сбор данных, анализ и классификация источников}},
  {Модуль 3, {\textbf{(2--3)} Применение комплинг \\ методов, обработка \\ результатов}},
  {Модуль 4, \textbf{(4)} Написание статей}}




\end{frame}

\section{Что уже сделано?}
\begin{frame}{Что уже сделано?}
\begin{itemize}
\item Создан репозиторий проекта на GitHub
\item 04.11 проведена первая беседа с научным руководителем, определена общая стратегия работы над проектом и план работы на 2 модуль
\item Участники проекта распределили между собой ресурсы для анализа, результаты исследования выложили на GitHub
\item Обсудили возникшие в ходе анализа ресурсов вопросы:
	\begin{itemize}
	\item Какие  тексты включать в исследование?
    \item Включать ли новостные статьи?
    \item Переводные статьи?
	\end{itemize}
\item Обратились за помощью к редакторам научно-популярных сайтов
\end{itemize}
\end{frame}


\section{Ресурсы}
\begin{frame}{N+1. Контент}
   \begin{enumerate}
   \item Новости (на основе \textit{Science}, \textit{Nature} и др. авторитетных изданий)
   \item Материалы
   		\begin{itemize}
   		\item Лонгриды (оригинальные статьи авторов)
        \item Экспертные интервью
   		\end{itemize}
    \item Блоги
    \item Тесты и игры, которые мы рассматривать не будем
   \end{enumerate}

\end{frame}

\begin{frame}{N+1. Структура}
\begin{itemize}
\item Тексты разбиты по рубрикам, разным предметным областям (около 35)
\item На главной странице разбиты по тегам, а не по рубрикам
\item Едины формат разметки текстов статей
\item Тегами в интервью обозначена прямая речь интервьюера и эксперта
\item \textbf{Проблема:} лонгриды и экспертные интервью никак не отмечены в URL, а просто лежат в разделе \textit{materials}. 
\end{itemize}

\end{frame}


\begin{frame}{ПостНаука. Форматы}
\begin{itemize}
	\item короткие видео-лекции с расшифровками
	\item FAQ — тексты, созданные на основе интервью, но оформленные как монологии
    \item talks — интервью «за жизнь» 
	\item longreads — статьи и отрывки из книг
	\item подборки книг и фильмов
    \item микроформаты — тесты, мультфильмы
\end{itemize}
\end{frame}

\begin{frame}{ПостНаука. Структура}
\begin{itemize}
	\item есть рубрикатор по URL
	\item есть разбивка по темам
    \item есть разбивка по "проектам" (главы отделенные от статей), но это не отображено в урлах
	\item почти все материалы — «прямая речь», в остальных случаях лишнее можно отсечь за счет специального оформление

\end{itemize}
\end{frame}

\begin{frame}{Элементы (?)}
\begin{itemize}
	\item много новостей
	\item но кто их автор — не всегда понятно
    \item очень много энциклопедических разделов — календарь, библиотека, "масштабы", и так далее
	\item большинство материалов —тексты без каких-либо цитат, отсылок, и комментариев

\end{itemize}
\end{frame}

\begin{frame}{Предварительный анализ источников}
\begin{enumerate}
   \item \textbf{8 источников} \\
   		N + 1, ProScience, Geektimes, Чердак, ПостНаука, Элементы, Полит.ру, Индикатор
	\item Контент сайтов
    	\begin{itemize}
   		\item Новости науки
        \item Longreads - оригинальные статьи
        \item Лекции
        \item Интервью с экспертами
        \item Переводы статей
        \end{itemize}
    \item Структура
    	\begin{itemize}
        \item Разная рубрикация
        \item Разные способы кодировки значимой информации (рубрики, экспертные комментарии ученых)
        \item Комментарии, рейтинги
        \end{itemize}
\end{enumerate}
\end{frame}

\begin{frame}{Вопросы}
\begin{itemize}
	\item Нужно ли использовать переводные статьи? \\Научные новости? \\Отрывки из книг?
    \item Как разграничивать новости и монологи ученых?
    \item Получение данных (редакторы, автоматические обращения к сайтам, API)
\end{itemize}
\end{frame}

\section{Дальнейшие действия}

\begin{frame}{Что дальше?}
\begin{figure}
\includegraphics[height=6cm]{zhdun.jpg}
\end{figure}
\textbf{Ждем текстов, учимся выкачивать сайты.}
	

\end{frame}

\begin{frame}{Спасибо за внимание!}
\begin{figure}
\includegraphics[height=6cm]{zhdun.jpg}
\end{figure}
\end{frame}



	

% \subsection{Tables and Figures}

% \begin{frame}{Tables and Figures}

% \begin{itemize}
% \item Use \texttt{tabular} for basic tables --- see Table~\ref{tab:widgets}, for example.
% \item You can upload a figure (JPEG, PNG or PDF) using the files menu. 
% \item To include it in your document, use the \texttt{includegraphics} command (see the comment below in the source code).
% \end{itemize}

% Commands to include a figure:
%\begin{figure}
%\includegraphics[width=\textwidth]{your-figure's-file-name}
%\caption{\label{fig:your-figure}Caption goes here.}
%\end{figure}

% \begin{table}
% \centering
% \begin{tabular}{l|r}
% Item & Quantity \\\hline
% Widgets & 42 \\
% Gadgets & 13
% \end{tabular}
% \caption{\label{tab:widgets}An example table.}
% \end{table}

% \end{frame}

% \subsection{Mathematics}

% \begin{frame}{Readable Mathematics}

% Let $X_1, X_2, \ldots, X_n$ be a sequence of independent and identically distributed random variables with $\text{E}[X_i] = \mu$ and $\text{Var}[X_i] = \sigma^2 < \infty$, and let
% $$S_n = \frac{X_1 + X_2 + \cdots + X_n}{n}
%       = \frac{1}{n}\sum_{i}^{n} X_i$$
% denote their mean. Then as $n$ approaches infinity, the random variables $\sqrt{n}(S_n - \mu)$ converge in distribution to a normal $\mathcal{N}(0, \sigma^2)$.

% \end{frame}

\end{document}
